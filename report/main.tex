\documentclass[czech,a4paper,11pt]{article}
%\documentclass[]{article}
\usepackage[pdftex]{graphicx}
\usepackage{graphicx}
\usepackage[czech]{babel}		
\usepackage[T1]{fontenc}	
\usepackage[utf8]{inputenc} 	
\newenvironment{bottompar}{\par\vspace*{\fill}}{\clearpage}
\usepackage{enumerate}
\usepackage{url}

\newcommand{\HRule}{\rule{\linewidth}{0.5mm}}

\begin{document}

\begin{titlepage}
\begin{center}

\includegraphics[width=0.15\textwidth]{./cvut-logo-bw.pdf}~\\[1cm]

\textsc{\LARGE České vysoké učení technické }\\[1.5cm]

\textsc{\Large MI-MKY úkol 3}\\[0.5cm]

% Title
\HRule \\[0.4cm]
{ \huge \bfseries Lámání DLP pomocí Babystep-Giantstep algoritmu \\[0.4cm] }

\HRule \\[1.5cm]

% Author and supervisor
\begin{minipage}{0.4\textwidth}
\begin{flushleft} \large
\emph{Autor:}\\
Tomáš \textsc{Sušánka}
\end{flushleft}
\end{minipage}

\vfill

% Bottom of the page
{\large \today}

\end{center}
\end{titlepage}


\newpage
\tableofcontents

\newpage
\section{Zadání a řešení}
Úkolem bylo implementovat algoritmus Babystep-Giantstep, který řeší problém diskrétního algoritmu.

Pro zadané $g(x)$ a mé soukromé $h(x)$ vyjde:

\begin{table}[h!]
  \begin{center}
	0x4658ca487
	$x^{34}+x^{30}+x^{29}+x^{26}+x^{24}+x^{23}+x^{19}+x^{18}+x^{15}+x^{13}+x^{10}+x^7+x^2+x+1$
  \end{center}
\end{table}


\section{Implementace}

Nejdříve jsem celý algoritmus implementoval pro celá čísla, abych si pořádně uvědomil, jak algoritmus funguje. Poté jsem implementoval násobení a sčítání v tělese $GF(2^{35})$\mbox{*}.

Násobení je implementováno pomocí algoritmu Double-and-Add. Pro vynásobení $A=Ax \bmod {f(x)}$ je použito LFSR (Linear feedback shift register). Oba tyto postupy jsme se učili v předmětu MI-BHW, a proto mi přišlo vhodné je použít.

Sčítání je pouze jednoduché využití operátoru XOR.

\section{Problémy}
Problémů jsem měl až nečekaně hodně. Při implementaci násobení a práci s bity se člověk velmi lehce sekne.

Největší problém se kterým jsem se setkal byla nízká efektivita programu. Poté co jsem odladil násobení i hlavní část programu, jsem dal oba celky dohromady a výpočet mi trval kolem 15 minut. Dlouhou dobu jsem nemohl objevit kde je zakopaný pes, stále jsem hledal chybu v implementaci násobení. Chyba samozřejmě byla úplně někde jinde.

Přestože jsem hodnoty v tabulce seřadil, hledal jsem v nich stále lineárně. Je však zřejmé, že v tomto případě je nasnadě využít binární půlení. Ze složitosti $O(n^2)$ jsem se náhle dostal na složitost $O(n\log{}n)$, což dělá dramatický rozdíl. Řešení po této opravě trvá 15 vteřin.


\section{Příloha}



\end{document}
